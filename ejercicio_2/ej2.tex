\documentclass[12pt,a4paper,english]{extarticle}
\usepackage[T1]{fontenc}
\usepackage[utf8]{inputenc}
\usepackage{fourier}
\usepackage{geometry}
\geometry{verbose,tmargin=2.5cm,bmargin=2cm,lmargin=2.5cm,rmargin=2cm}
\usepackage{float}
\usepackage{textcomp}
\usepackage{amsmath}
\usepackage{stackrel}
\usepackage{graphicx}
\usepackage{esint}
\usepackage{tikz}
\usetikzlibrary{matrix,calc}

\makeatletter

\providecommand{\tabularnewline}{\\}

\usepackage{fancyhdr}
\usepackage{lscape}
\usepackage{amssymb}
\pagestyle{fancy}
\lhead{Electronica III - 22.13}
\chead{TPL1}
\rhead{ITBA}
\renewcommand{\headrulewidth}{1pt}
\renewcommand{\footrulewidth}{1pt}

\makeatother

\usepackage[english]{babel}
\input{BaseKarnaugh}
\begin{document}

\section*{Ejercicio 2}
Se parte de la siguiente función expresada en maxtérminos:
\[
    f(d,c,b,a)=\prod{(M_{0},M_{1},M_{5},M_{7},M_{8},M_{10},M_{14},M_{15})}    
\]

Tomando como variables de entrada lógicas a $d,c,b,a$. Por simplicidad,
 se expresa la misma función en minitérminos para operar luego:
\[
    f(d,c,b,a)=\sum{(m_{2},m_{3},m_{4},m_{6},m_{9},m_{11},m_{12},m_{13})}    
\]

A partir de esta, construimos la función sin simplificar:
\[
    f(d,c,b,a)=(\overline{d} \cdot \overline{c} \cdot b \cdot \overline{a})+
    (\overline{d} \cdot \overline{c} \cdot b \cdot a)+
    (\overline{d} \cdot c \cdot \overline{b} \cdot \overline{a})+
    (\overline{d} \cdot c \cdot b \cdot \overline{a})+
    (d \cdot \overline{c} \cdot \overline{b} \cdot a)+
    (d \cdot \overline{c} \cdot b \cdot a)+
    (d \cdot c \cdot \overline{b} \cdot \overline{a})+
    (d \cdot c \cdot \overline{b} \cdot a)           
\]

Agrupamos por factor común en forma conveniente:
\begin{eqnarray}
    \nonumber f(d,c,b,a)&=&\underbrace{(\overline{d} \cdot \overline{c} \cdot b \cdot \overline{a})+
    (\overline{d} \cdot \overline{c} \cdot b \cdot a)}+\underbrace{
    (\overline{d} \cdot c \cdot \overline{b} \cdot \overline{a})+
    (\overline{d} \cdot c \cdot b \cdot \overline{a})}+\underbrace{
    (d \cdot \overline{c} \cdot \overline{b} \cdot a)+
    (d \cdot \overline{c} \cdot b \cdot a)}+\\
    \nonumber &\longrightarrow&\underbrace{(d \cdot c \cdot \overline{b} \cdot \overline{a})+
    (d \cdot c \cdot \overline{b} \cdot a)}    
\end{eqnarray}
\[
    f(d,c,b,a)=[\overline{d} \cdot \overline{c} \cdot b \cdot \underbrace{(\overline{a}+a)}_1]+
    [\overline{d} \cdot c \cdot \overline{a} \cdot \underbrace{\overline{b}+b)}_1]+
    [d \cdot \overline{c} \cdot a \cdot  \underbrace{(\overline{b}+b)}_1]+
    [d \cdot c \cdot \overline{b} \cdot \underbrace{(\overline{a}+a)}_1]       
\]
\[
    \boxed{f(d,c,b,a)=(\overline{d} \cdot \overline{c} \cdot b)+
    (\overline{d} \cdot c \cdot \overline{a})+
    (d \cdot c \cdot \overline{b})+  
    (d \cdot \overline{c} \cdot a)}     
\]
\\ %Salto de linea
Análogamente, a partir de la expresión en miniterminos reducimos la 
función mediante un mapa de Karnaugh:

\begin{centering}
    \begin{Karnaugh}
        \minterms{2,3,4,6,9,11,12,13}
        \maxterms{0,1,5,7,8,10,14,15}
        \implicant{3}{2}{red}
        \implicantcostats{4}{6}{red}
        \implicant{12}{13}{red}
        \implicant{9}{11}{red}
    \end{Karnaugh}
\par\end{centering}

Del primer grupo (primer fila) se tene que $d$, $c$ y $b$ quedan 
constantes, por lo que el primer factor queda de la forma 
$ \overline{d} \overline{c} b$.\par
Del segundo grupo (segunda fila) se tiene que $d$, $c$ y $a$ son 
constantes, por lo que dicho factor queda de la forma $ \overline{d} c \overline{a}$.\par
Del tercer grupo (tercer fila) quedan constantes $d$, $c$ y $b$, por lo 
que este factor queda de la forma $d c \overline{b}$.\par
Finalmente, de la última fila, en el grupo se mantienen constantes
$d$, $c$ y $a$, por lo que este último factor queda de la forma 
$d \overline{c} a$.\par
Sumando los términos parciales se obtiene la función buscada:
\[
    \boxed{f(d,c,b,a)=(\overline{d} \cdot \overline{c} \cdot b)+
    (\overline{d} \cdot c \cdot \overline{a})+
    (d \cdot c \cdot \overline{b})+  
    (d \cdot \overline{c} \cdot a)}     
\]
Verificando asi que se llega a la misma expresión.


\end{document}